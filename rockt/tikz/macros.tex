\usetikzlibrary{matrix, calc, arrows, decorations.markings}

% relative positioning
\newcommand{\at}[3]{
  \begin{scope}[shift={(#1,#2)}]
    #3
  \end{scope} 
}

% scaling
\newcommand{\scale}[2]{
  % \begin{scope}[transform canvas={scale=#1}]
  \begin{scope}[scale=#1]
    #2
  \end{scope}
}

% naming
\newcommand{\name}[2]{
  \begin{scope}[local bounding box=#1]
    #2
  \end{scope}
}

% shortening both ends of an arrow
\tikzset{ shorten <>/.style={ shorten >=#1, shorten <=#1 } }

% arrows
\tikzstyle{arrow}=[draw, -latex, thick] 
\tikzstyle{axis}=[draw, -stealth, thick]
\tikzstyle{vector}=[arrow, very thick]

% coordinate systems
\newcommand{\coordtwo}{
  \node (0) at (0, 0) {};
  \node (1) at (0, 3) {};
  \node (2) at (3, 0) {};
  \draw[axis] (0.center) to (1);
  \draw[axis] (0.center) to (2);
}

\newcommand{\coordtwogrid}{
  \coordtwo
  \draw[help lines] (0,0) grid (2.5,2.5);
}

\newcommand{\coordthree}{
  \coordtwo
  \node (3) at (1, 1.5) {};
  \draw[axis] (0.center) to (3);
}

% \getwidthofnode will measure the width of the node given as its second
% parameter and store it into the first parameter.
\makeatletter
\newcommand\getwidthofnode[2]{%
    \pgfextractx{#1}{\pgfpointanchor{#2}{east}}%
    \pgfextractx{\pgf@xa}{\pgfpointanchor{#2}{west}}% \pgf@xa is a length defined by PGF for temporary storage. No need to create a new temporary length.
    \addtolength{#1}{-\pgf@xa}%
}
\makeatother