%\NeedsTeXFormat{LaTeX2e}[1994/06/01]
%\ProvidesPackage{tikztensor}[2014/07/25 tikztensor]
%\RequirePackage{tikz}

\usepackage{tikz}
\usetikzlibrary{calc,trees,positioning,arrows,chains,shapes.geometric,%
  decorations.pathreplacing,decorations.pathmorphing,shapes,%
  matrix,shapes.symbols,fit,decorations}

\usepackage{xcolor}
\usepackage{xparse}
%\usetikzlibrary{matrix, calc, arrows, decorations.markings}

% relative positioning
\newcommand{\at}[3]{
  \begin{scope}[shift={(#1,#2)}]
    #3
  \end{scope} 
}

% scaling
\newcommand{\scale}[2]{
  % \begin{scope}[transform canvas={scale=#1}]
  \begin{scope}[scale=#1]
    #2
  \end{scope}
}

% naming
\newcommand{\name}[2]{
  \begin{scope}[local bounding box=#1]
    #2
  \end{scope}
}

% hiding
\newcommand{\hide}[2]{
  \begin{scope}[onslide={<#1>{every node/.append style={opacity=0.1}, every path/.append style={opacity=0.1}}}]
    #2
  \end{scope}
}

% shortening both ends of an arrow
\tikzset{ shorten <>/.style={ shorten >=#1, shorten <=#1 } }

% arrows
\tikzstyle{arrow}=[draw, -latex, thick] 
\tikzstyle{axis}=[draw, -stealth, thick]
\tikzstyle{vector}=[arrow, very thick]
\tikzstyle{mapping}=[draw, -open triangle 45, very thick, color=hidden]

% coordinate systems
\newcommand{\coordtwo}{
  \node (0) at (0, 0) {};
  \node (1) at (0, 3) {};
  \node (2) at (3, 0) {};
  \draw[axis] (0.center) to (1);
  \draw[axis] (0.center) to (2);
}

\newcommand{\coordtwogrid}{
  \coordtwo
  \draw[help lines] (0,0) grid (2.5,2.5);
}

\newcommand{\coordthree}{
  \coordtwo
  \node (3) at (1, 1.5) {};
  \draw[axis] (0.center) to (3);
}

% \getwidthofnode will measure the width of the node given as its second
% parameter and store it into the first parameter.
\makeatletter
\newcommand\getwidthofnode[2]{%
  \pgfextractx{#1}{\pgfpointanchor{#2}{east}}%
  \pgfextractx{\pgf@xa}{\pgfpointanchor{#2}{west}}% \pgf@xa is a length defined by PGF for temporary storage. No need to create a new temporary length.
  \addtolength{#1}{-\pgf@xa}%
}

% beamer util
\tikzset{onslide/.code args={<#1>#2}{%
  \only<#1>{\pgfkeysalso{#2}} % \pgfkeysalso doesn't change the path
}}

\tikzset{temporal/.code args={<#1>#2#3#4}{%
  \temporal<#1>{\pgfkeysalso{#2}}{\pgfkeysalso{#3}}{\pgfkeysalso{#4}} % \pgfkeysalso doesn't change the path
}}

%% document
\newcommand{\doc}{
  \node[draw, thick, minimum width=1cm, minimum height=1.4cm, fill=white] at (0,0) {};
  \foreach \y in {-.5, -.3, -.1, .1, .3, .5} {
    \draw[thick] (-.4, \y) -- (.4, \y);
  }
}

% corpus
\newcommand{\corpus}{
  \foreach \x in {.2, .1, 0} {
    \at{\x}{\x}{\doc};
  }
}

% database
\tikzset{%
  database/.style={
    cylinder,
    cylinder uses custom fill,
    cylinder body fill=black!10,
    cylinder end fill=black!10,
    shape border rotate=90,
    aspect=0.25,
    thick,
    draw
  }
}

% user
\newcommand{\user}{
  \node (0) at (0, 0.5) {};
  \node (1) at (0, -0.5) {};
  \node (2) at (-0.5, -1.25) {};
  \node (3) at (0.5, -1.25) {};
  \node (4) at (0.75, 0) {};
  \node (5) at (-0.75, 0) {};
  \node (6) at (0, 1) {};
  \draw[thick] (0.center) to (1.center);
  \draw[thick] (1.center) to (2.center);
  \draw[thick] (1.center) to (3.center);
  \draw[thick] (5.center) to (4.center);
  \draw[thick, fill = white] (0,.9) circle (.4);
}  
%\usepackage{xcolor}
%general
\colorlet{dark-blue}{blue!50!black}
\colorlet{dark-purple}{purple!50!black}
\colorlet{dark-red}{red!50!black}
\colorlet{dark-green}{green!50!black}
%package-specific
\colorlet{todo}{red!85!black}
\colorlet{todoref}{purple!70!black}

  


\newcommand{\tikztensorx}{1}
\newcommand{\tikztensory}{1}
\newcommand{\tikztensorz}{1}

\newcommand{\setcoordinates}[3]{
  \renewcommand{\tikztensorx}{#1}
  \renewcommand{\tikztensory}{#2}
  \renewcommand{\tikztensorz}{#3}

  \coordinate (0) at (0,0,0);
  \coordinate (x) at (\tikztensorx,0,0);
  \coordinate (y) at (0,\tikztensory,0);
  \coordinate (z) at (0,0,-\tikztensorz);
  \coordinate (xy) at (\tikztensorx,\tikztensory,0);
  \coordinate (xz) at (\tikztensorx,0,-\tikztensorz);
  \coordinate (yz) at (0,\tikztensory,-\tikztensorz);
  \coordinate (xyz) at (\tikztensorx,\tikztensory,-\tikztensorz); 
}

\newcommand{\debugcoordinates}{
  \foreach \xy in {0, x, y, z, xy, xz, yz, xyz}{
    \node at (\xy) {\xy};
  }
}


\newcommand{\tensorback}[1]{
  \name{bottom}{\draw[black, fill = black!5, #1] (0) -- (x) -- (xz) -- (z) -- cycle;}
  \name{back}{\draw[black, fill = black!5, #1] (z) -- (xz) -- (xyz) -- (yz) -- cycle;}
  \name{left}{\draw[black, fill = black!5, #1] (0) -- (z) -- (yz) -- (y) -- cycle;}
}

\DeclareDocumentCommand{\tensorfront}{O{1} m}{
  % right
  \draw[black, fill = black, fill opacity=0.15, #2] (x) -- (xz) -- (xyz) -- (xy) -- cycle;
  % top
  \draw[black, fill = black, fill opacity=0.1, #2] (y) -- (xy) -- (xyz) -- (yz) -- cycle;
  % front
  \ifnum #1=1
  \draw[black, fill = black, fill opacity=0.05, #2] (0) -- (x) -- (xy) -- (y) -- cycle;
  \fi
}

\DeclareDocumentCommand{\tensor}{O{} O{} O{1} O{1} O{1} O{1} m m m}{
  \name{lhs}{
   % \setcoordinates{#3 * 0.25}{#4 * 0.25}{#5 * 0.25}
   % \tensorback{#1}

    #2
    
    \setcoordinates{#7 * 0.25}{#8 * 0.25}{#9 * 0.25}
    \tensorfront[#6]{#1}
    \tensorgrid[#3][#4][#5]{#7}{#8}{#9}
  }
}


\DeclareDocumentCommand{\tensorgrid}{O{1} O{1} O{1} m m m}{
    %slice vertically
    \ifnum #1=1
    \foreach \i in {0,...,#4} {
      \draw[color=white] (\i*0.25,0) -- (\i*0.25,#5*0.25);
      \draw[color=white] (\i*0.25,#5*0.25) -- (\i*0.25,#5*0.25,-#6*0.25); 
    }
    \fi
    
    %slice horizontally
    \ifnum #2=1
    \foreach \i in {0,...,#5} {
      \draw[color=white] (0,\i*0.25) -- (#4*0.25,\i*0.25);
      \draw[color=white] (#4*0.25,\i*0.25) -- (#4*0.25,\i*0.25,-#6*0.25); 
    }
    \fi

    %slice laterally
    \ifnum #3=1
    \foreach \i in {0,...,#6} {
      \draw[color=white] (0,#5*0.25,-\i*0.25) -- (#4*0.25,#5*0.25,-\i*0.25);
      \draw[color=white] (#4*0.25,0,-\i*0.25) -- (#4*0.25,#5*0.25,-\i*0.25);
    }
    \fi
}


\DeclareDocumentCommand{\tensorop}{O{.5em} O{1.5em} m}{
  \node[right = #1 of lhs.south east, yshift=#2] (op) {#3};
  \node[right = #1 of 0 -| op] {};
  \pgfgetlastxy{\nextx}{\nexty};
}

\DeclareDocumentCommand{\calculatenext}{O{.5em} O{2em} m}{
  \node[right = #1 of #3.south east, yshift=#2] (tmp) {};
  \pgfgetlastxy{\nextx}{\nexty};
}

\DeclareDocumentCommand{\setnext}{O{} m}{
  \node[#1] at (#2) {};
  \pgfgetlastxy{\nextx}{\nexty};
}

\DeclareDocumentCommand{\labelaxis}{O{black} m m m m}{
  \draw[ultra thick, #1] (#2) -- node[midway, #4] {#5} (#3);
}

\DeclareDocumentCommand{\labelx}{O{dark-green} m}{
  \labelaxis[#1]{0}{x}{below}{#2}
}

\DeclareDocumentCommand{\labely}{O{red} m}{
  \labelaxis[#1]{0}{y}{left}{#2}
}

\DeclareDocumentCommand{\labelz}{O{blue} m}{
  \labelaxis[#1]{y}{yz}{above left}{#2}
}


\DeclareDocumentCommand{\tensorshift}{O{0} O{0} O{0} m}{
  \begin{scope}[shift={(#1 * 0.25,#2 * 0.25,-#3 * 0.25)}]
    #4
  \end{scope} 
}



%\endinput
